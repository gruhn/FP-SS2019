\documentclass[a4paper,12pt,oneside]{book}

\usepackage[ngerman]{babel}
\usepackage{color}
\usepackage{amsmath}
\usepackage{amsfonts}
\usepackage{fancyhdr}
\usepackage{fancyvrb}
\DefineVerbatimEnvironment{code}{Verbatim}{fontsize=\small}
\DefineVerbatimEnvironment{example}{Verbatim}{fontsize=\small}
\usepackage{listings}
\newcommand{\ignore}[1]{}

\usepackage{tikz}
\usetikzlibrary{shapes}
\usetikzlibrary{positioning}
\tikzstyle{data}=[draw,text centered]
\usetikzlibrary{arrows,positioning, calc}
\tikzstyle{vertex}=[draw,circle,minimum size=18pt,inner sep=0pt]

\fboxrule0.3mm
\definecolor{rahmen}{gray}{.3}      % Dunkelgrauer Rahmen
\definecolor{grund}{gray}{.85}         % Hellgrauer Hintergrund

\lstset{ 
    language=Haskell, % choose the language of the code
    basicstyle=\fontfamily{pcr}\selectfont\footnotesize\color{black},
    keywordstyle=\color{black}\mdseries, % style for keywords
    numbers=none, % where to put the line-numbers
    numberstyle=\tiny, % the size of the fonts that are used for the line-numbers     
    backgroundcolor=\color{grund},
    showspaces=false, % show spaces adding particular underscores
    showstringspaces=false, % underline spaces within strings
    showtabs=false, % show tabs within strings adding particular underscores
    frame=single, % adds a frame around the code
    tabsize=2, % sets default tabsize to 2 spaces
    rulesepcolor=\color{rahmen},
    rulecolor=\color{rahmen},
    captionpos=b, % sets the caption-position to bottom
    breaklines=true, % sets automatic line breaking
    breakatwhitespace=false, 
}
 
\pagestyle{fancy}
\usepackage[left=30mm,right=30mm, top=27mm, bottom=22mm]{geometry}

\fancyhf{}
\rfoot{ \textbf{Page \thepage}}
\lhead{\begin{footnotesize}Functional Programming SS 2019\end{footnotesize}}
\rhead{\begin{footnotesize}379455, 402403, 389343, 402372\end{footnotesize}}
\lfoot{\begin{footnotesize} 
\end{footnotesize}}
\renewcommand{\headrulewidth}{1pt}
\renewcommand{\footrulewidth}{1 pt} 

\begin{document}
\setlength{\parindent}{0em} 


\begin{center} 
\textbf{\huge{Functional Programming} \\ \large{ Excercise Sheet 3}} % Nummerierung anpassen

~\\
Emilie Hastrup-Kiil (379455), 
Julian Schacht (402403), \\
Niklas Gruhn (389343), 
Maximilian Loose (402372)
\end{center}
\textbf{Excercise 1} \\ % Nummerierung anpassen
a)
\begin{lstlisting}
collatz :: Int -> [Int] 
collatz x = iterate (\y -> if mod y 2 == 0 then div y 2 else 3*y+1) x 
    
    
total_stopping_time :: Int -> Int
total_stopping_time 1 = 3
total_stopping_time x = length (takeWhile (\y -> y /= 1) (collatz x))
\end{lstlisting}
~\\
b)
\begin{lstlisting}
check_collatz :: Int -> Bool
check_collatz 1 = False
check_collatz n = elem 1 (take n (collatz n))
\end{lstlisting}
~\\
\textbf{Excercise 2} \\% Nummerierung anpassen
a)
\begin{lstlisting}
drop_mult :: Int -> [Int] -> [Int]
drop_mult x xs = [y | y <- xs , mod y x /= 0]

dropall :: [Int]-> [Int]
dropall (x:xs) = x : dropall (drop_mult x xs)

primes :: [Int]
primes = dropall [2 ..]

goldbach :: Int -> [(Int,Int)]
goldbach n = [(x,y)| mod n 2 == 0, x<-takeWhile (<n) primes, y<-filter (>=x) (takeWhile (<n) primes), x+y == n]
\end{lstlisting}
~\\
b)
\begin{lstlisting}
range :: [a] -> Int -> Int -> [a]
range xs a b = [xs !! i| i <- [a .. b]] 
\end{lstlisting}
~\\

\textbf{Excercise 3} \\% Nummerierung anpassen
~\\     
\textbf{Excercise 4} \\% Nummerierung anpassen
a)
\begin{center}
\begin{tikzpicture}[node distance=1cm and 1cm]

    \node []  							(A)    {((-1,False),0) = $y_1$};
    \node [below=1cm of A]  				(B)    {((-1,$\bot$),0)};
    \node [left=2.5cm of B]  			       	(C)    {(($\bot$,False),0)};
    \node [right=2.5cm of B]  				(D)    {((-1,False),$\bot$)};
    \node [below=1cm of B]  				(E)    {(($\bot$,False),$\bot$)};
    \node [left=2.5cm of E]  				(F)    {((-1,$\bot$),$\bot$)};
    \node [right=2.5cm of E]  				(G)    {(($\bot$,$\bot$),0)};
    \node [below=1cm of E]  				(H)    {(($\bot$,$\bot$),$\bot$)};
    
 \path[<-]
    (A) edge node {} (B)
    (A) edge node {} (C)
    (A) edge node {} (D)
    (B) edge node {} (F)
    (B) edge node {} (G)
    (C) edge node {} (E)
    (C) edge node {} (F)
    (D) edge node {} (E)
    (D) edge node {} (G)
    (E) edge node {} (H)
    (F) edge node {} (H)
    (G) edge node {} (H);

    
\end{tikzpicture}
\end{center}

All elements $x \in (( \mathbb{Z}_{\bot} \times \mathbb{B}_{\bot}) \times \mathbb{Z}_{\bot}$ that are less defined than $y_1$ are given by $((-1,\bot),0),((\bot,False),0),((-1,False),\bot),((\bot,False),\bot),((-1,\bot),\bot),((\bot,\bot),0)$ and the smallest Element $((\bot,\bot),\bot)$.
\\

\begin{center}
\begin{tikzpicture}[node distance=1cm and 1cm]

    
    \node[] at (0, 4)  (A)  {((-1,$\bot$),2) = $y_2$};
    \node[] at (3, 2)   (B) {(($\bot$,$\bot$),2)};
    \node[] at (-3, 2) (C)  {((-1,$\bot$),$\bot$)};
    \node[] at (0, 0)  (D)  {(($\bot$,$\bot$),$\bot$)};
    
 \path[<-]
    (A) edge node {} (B)
    (A) edge node {} (C)
    (C) edge node {} (D)
    (B) edge node {} (D);

    
\end{tikzpicture}
\end{center}
All elements $x \in (( \mathbb{Z}_{\bot} \times \mathbb{B}_{\bot}) \times \mathbb{Z}_{\bot}$ that are less defined than $y_2$ are given by $((\bot,\bot),2),((-1,\bot),\bot)$ and the smallest Element $((\bot,\bot),\bot)$.
\newpage
b)�\\
Given a domain $D= \underbrace{\mathbb{Z}_{\bot} \times ... \times \mathbb{Z}_{\bot}}_{n\text{\ times}}$ for $0<n \in \mathbb{N}$ and a chain $S \subseteq D$ then $\sup \{\vert S \vert \text{ } \vert  \text{ }  S \subseteq D, \text{S is a chain}\} = n+1$.\\ \\
We use induction on n to show that this holds for all $n \in \mathbb{N}, n>0$.\\

\textbf{Base case}. $n = 1$\\
Let $D=\mathbb{Z}_{\bot}$, then $\bot \in D$ and for all $x \in \mathbb{Z}: x \in \mathbb{Z}_{\bot}$. For $x,y \in S$ with $x,y \in \mathbb{Z}$ it follows from the definition of S that either $ x \sqsubseteq y$ or $y \sqsubseteq x$. By the definition of $\sqsubseteq$ and x and y, this is only true iff $x=y$. As a result,  $\vert \{x \in \mathbb{Z} \text{ } \vert  \text{ }  x \in  S\text{ with }S \subseteq D, \text{S is a chain}\} \vert \leq 1$.\\
However, $\bot \sqsubseteq x$ for all $x \in \mathbb{Z}$ hence $\sup \{\vert S \vert \text{ } \vert  \text{ }  S \subseteq D, \text{S is a chain}\} = 1+1 = 2 = n+1$.\\ \\

\textbf{Induction hypothesis} \\
$\sup \{\vert S \vert \text{ } \vert  \text{ }  S \subseteq D, \text{S is a chain}\} = n+1$ holds for $n \in \mathbb{N}, 0<n$ with $D= \underbrace{\mathbb{Z}_{\bot} \times ... \times \mathbb{Z}_{\bot}}_{n\text{\ times}}$.\\

\textbf{Induction step} $ n\rightarrow n+1$\\
Let $D=\underbrace{\mathbb{Z}_{\bot} \times ... \times \mathbb{Z}_{\bot}}_{n+1\text{\ times}}$. \\
It follows from the hypothesis that if $D_n=\underbrace{\mathbb{Z}_{\bot} \times ... \times \mathbb{Z}_{\bot}}_{n\text{\ times}}$ then $\sup \{\vert S_n \vert \text{ } \vert  \text{ }  S_n \subseteq D_n, S_n\text{ is a chain}\} = n+1$. \\ 
\\ Now we extend each n-tuple $ (n_1,n_2, ...,n_n)$ in the chain $S_n$ ($n_1, n_2,...,n_n \in \mathbb{Z}_{\bot}$) so that we get (n+1)-tuples $(n_1,n_2, ...,n_n, x)�\in D, x�\in \mathbb{Z}_{\bot}$.\\
\\
Case 1: $x\in\mathbb{Z}$\\
For two tuples $t_i = (n_{i_1},n_{i_2}, ...,n_{i_n}), t_j = (n_{j_1},n_{j_2}, ...,n_{j_n}) \in S_n$ either $t_i \sqsubseteq t_j$ or  $t_j \sqsubseteq t_i$ applies. Then also $(n_{i_1},n_{i_2}, ...,n_{i_n},x) \sqsubseteq (n_{j_1},n_{j_2}, ...,n_{j_n},x)$ or  $(n_{j_1},n_{j_2}, ...,n_{j_n},x) \sqsubseteq (n_{i_1},n_{i_2}, ...,n_{i_n},x)$ since in particular $x\sqsubseteq x$ and consequently $(n_{j_1},n_{j_2}, ...,n_{j_n},x)$, $(n_{i_1},n_{i_2}, ...,n_{i_n},x) \in S$ with $\sup \{\vert S \vert \text{ } \vert  \text{ }  S \subseteq D, \text{S is a chain}\} \geq \sup \{\vert S_n \vert \text{ } \vert  \text{ }  S_n \subseteq D_n, S_n\text{ is a chain}\} = n+1$. \\
If $(n_1,n_2, ...,n_n, x)�\in S$ and $(n_1,n_2, ...,n_n, y)�\in S$ then $x=y$ because neither $x \sqsubseteq y$ nor $y \sqsubseteq x$ applies. Similarily, If $t_i= (n_1,n_2, ...,n_n, x)�\in S$ and $t_j =(n_1,n_2,...,x,...,n_n)�\in S$ then neither $t_i \sqsubseteq t_j$ nor $t_j \sqsubseteq t_i$ applies, because $t_i \neq t_j$ and both are equally defined. This leads on to our second case.\\\\
Case 2: $(n_1,n_2, ...,n_n, x), n_1 = n_2 =... =n_n=x=\bot$\\
Then clearly $(n_1,n_2, ...,n_n, x) \in S$ since it is the smallest element of D and therefore $(n_1,n_2, ...,n_n, x) \sqsubseteq y$ for all $y\in D$. Resulting from this, $\sup \{\vert S \vert \text{ } \vert  \text{ }  S \subseteq D, \text{S is a chain}\} = (n+1)+1 = n+2$. The equation holds.
\hfill $\square$
\end{document}
