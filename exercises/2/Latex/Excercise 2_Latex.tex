\documentclass[a4paper,12pt,oneside]{book}

\usepackage[ngerman]{babel}
\usepackage{color}
\usepackage{fancyhdr}
\usepackage{fancyvrb}
\DefineVerbatimEnvironment{code}{Verbatim}{fontsize=\small}
\DefineVerbatimEnvironment{example}{Verbatim}{fontsize=\small}
\usepackage{listings}
\newcommand{\ignore}[1]{}

\fboxrule0.3mm
\definecolor{rahmen}{gray}{.3}      % Dunkelgrauer Rahmen
\definecolor{grund}{gray}{.85}         % Hellgrauer Hintergrund

\lstset{ 
    language=Haskell, % choose the language of the code
    basicstyle=\fontfamily{pcr}\selectfont\footnotesize\color{black},
    keywordstyle=\color{black}\mdseries, % style for keywords
    numbers=none, % where to put the line-numbers
    numberstyle=\tiny, % the size of the fonts that are used for the line-numbers     
    backgroundcolor=\color{grund},
    showspaces=false, % show spaces adding particular underscores
    showstringspaces=false, % underline spaces within strings
    showtabs=false, % show tabs within strings adding particular underscores
    frame=single, % adds a frame around the code
    tabsize=2, % sets default tabsize to 2 spaces
    rulesepcolor=\color{rahmen},
    rulecolor=\color{rahmen},
    captionpos=b, % sets the caption-position to bottom
    breaklines=true, % sets automatic line breaking
    breakatwhitespace=false, 
}
 
\pagestyle{fancy}
\usepackage[left=30mm,right=30mm, top=27mm, bottom=22mm]{geometry}

\fancyhf{}
\rfoot{ \textbf{Page \thepage}}
\lhead{\begin{footnotesize}Functional Programming SS 2019\end{footnotesize}}
\rhead{\begin{footnotesize}379455, 402403, 389343, 402372\end{footnotesize}}
\lfoot{\begin{footnotesize} 
\end{footnotesize}}
\renewcommand{\headrulewidth}{1pt}
\renewcommand{\footrulewidth}{1 pt} 

\begin{document}
\setlength{\parindent}{0em} 


\begin{center} 
\textbf{\huge{Functional Programming} \\ \large{ Excercise Sheet 2}} % Nummerierung anpassen

~\\
Emilie Hastrup-Kiil (379455), 
Julian Schacht (402403), \\
Niklas Gruhn (389343), 
Maximilian Loose (402372)
\end{center}
\textbf{Excercise 1} \\ % Nummerierung anpassen
a)
\begin{lstlisting}
data PriorityQueue a = Push a Int (PriorityQueue a) | EmptyQueue deriving Show

p :: PriorityQueue Int
p = Push 11 1 (Push 5 3 (Push 5 0 (Push 9 (-1) (Push 7 3 (Push 8 (-3) EmptyQueue)))))
\end{lstlisting}
~\\
b)
\begin{lstlisting}
isWaiting :: Eq a => a -> PriorityQueue a -> Bool
isWaiting _ EmptyQueue               = False
isWaiting x (Push v _ n) | x == v    = True
                         | otherwise = isWaiting x n
\end{lstlisting}
~\\           
c)
\begin{lstlisting}
fromList :: [(a,Int)] -> PriorityQueue a
fromList []           = EmptyQueue
fromList ((x,p):xs)   = Push x p (fromList xs)
\end{lstlisting}
~\\
d)
\begin{lstlisting}
-- auxiliary functions
delete :: PriorityQueue a -> Int -> PriorityQueue a
delete EmptyQueue _   = EmptyQueue
delete (Push v p n) x = if x==p then n else (Push v p (delete n x)) 
findElement :: PriorityQueue a -> Int -> a
findElement (Push v p n) x = if x==p then v else findElement n x

highestPriority :: PriorityQueue a -> Int
highestPriority EmptyQueue   = minBound
highestPriority (Push v p n) = max p (highestPriority n)

--main function
pop :: PriorityQueue a -> (a,PriorityQueue a)
pop x = (findElement x h, delete x h)
    where h = highestPriority x
\end{lstlisting}
~\\
e)
\begin{lstlisting}
toList :: PriorityQueue a -> [a]
toList EmptyQueue = []
toList q = x : toList y
   where (x,y) = pop q
\end{lstlisting}
\textbf{Excercise 2} \\% Nummerierung anpassen
a)
\begin{lstlisting}
data List a = Nil | Cons a (List a)
  deriving Show

instance Eq a => Eq (List a) where
  (==) Nil Nil = True
  (==) (Cons x xs) (Cons y ys)
    | x /= y = False
    | otherwise = xs == ys
  (==) _ _ = False
\end{lstlisting}
~\\     
b)
\begin{lstlisting}
class Eq a => Mono a where
  binOp :: a -> a -> a
  one :: a
  pow :: Word -> a -> a
  pow 0 _ = one
  pow n x = binOp x (pow (n-1) x)
\end{lstlisting}
~\\     
c)
\begin{lstlisting}
instance Mono Integer where
  binOp x y = x * y
  one = 1

instance Eq a => Mono (List a) where
  binOp (Cons x xs) ys = Cons x (binOp xs ys)
  binOp Nil ys = ys
  one = Nil
  \end{lstlisting}
~\\     
d)
\begin{lstlisting}
multiply :: Mono a => [(Word, a)] -> a
multiply [] = one
multiply ((n,x):xs) = binOp (pow n x) (multiply xs)
\end{lstlisting}
~\\     

\textbf{Excercise 3} \\% Nummerierung anpassen
a) 
\begin{lstlisting}
removeDuplicates :: Eq a => [a] -> [a]
removeDuplicates xs = foldr dropAll xs xs
  where
    dropAll x ys = x : filter (/=x) ys
\end{lstlisting}
~\\     
b)
\begin{lstlisting}
differentDigits :: Int -> Int
differentDigits number = foldr count 0 uniqueDigits
  where
    uniqueDigits = removeDuplicates (show number)
    count _ n = n+1
\end{lstlisting}
~\\     

\textbf{Excercise 4} \\% Nummerierung anpassen
a)
\begin{lstlisting}
data Polynomial a = Coeff a Int (Polynomial a)| Null deriving Show

q :: Polynomial Int
q = Coeff 4 3 (Coeff 2 1 (Coeff 5 0 Null))

foldPoly :: (a -> Int -> b -> b) -> b -> Polynomial a -> b
foldPoly f d Null = d
foldPoly f d (Coeff a b c) = f a b (foldPoly f d c)
\end{lstlisting}
~\\    
b)
\begin{lstlisting}
degree :: Polynomial Int -> Int
degree x = foldPoly (\ c n m -> if n > m then n else m) minBound x 
\end{lstlisting}
~\\    


\end{document}
