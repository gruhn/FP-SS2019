\documentclass[a4paper,12pt,oneside]{book}

\usepackage[ngerman]{babel}
\usepackage{color}
\usepackage{amsmath}
\usepackage{amsfonts}
\usepackage{fancyhdr}
\usepackage{fancyvrb}
\DefineVerbatimEnvironment{code}{Verbatim}{fontsize=\small}
\DefineVerbatimEnvironment{example}{Verbatim}{fontsize=\small}
\usepackage{listings}
\newcommand{\ignore}[1]{}

\usepackage{tikz}
\usetikzlibrary{shapes}
\usetikzlibrary{positioning}
\tikzstyle{data}=[draw,text centered]
\usetikzlibrary{arrows,positioning, calc}
\tikzstyle{vertex}=[draw,circle,minimum size=18pt,inner sep=0pt]

\fboxrule0.3mm
\definecolor{rahmen}{gray}{.3}      % Dunkelgrauer Rahmen
\definecolor{grund}{gray}{.85}         % Hellgrauer Hintergrund

\lstset{ 
    language=Haskell, % choose the language of the code
    basicstyle=\fontfamily{pcr}\selectfont\footnotesize\color{black},
    keywordstyle=\color{black}\mdseries, % style for keywords
    numbers=none, % where to put the line-numbers
    numberstyle=\tiny, % the size of the fonts that are used for the line-numbers     
    backgroundcolor=\color{grund},
    showspaces=false, % show spaces adding particular underscores
    showstringspaces=false, % underline spaces within strings
    showtabs=false, % show tabs within strings adding particular underscores
    frame=single, % adds a frame around the code
    tabsize=2, % sets default tabsize to 2 spaces
    rulesepcolor=\color{rahmen},
    rulecolor=\color{rahmen},
    captionpos=b, % sets the caption-position to bottom
    breaklines=true, % sets automatic line breaking
    breakatwhitespace=false, 
}
 
\pagestyle{fancy}
\usepackage[left=30mm,right=30mm, top=27mm, bottom=22mm]{geometry}

\fancyhf{}
\rfoot{ \textbf{Page \thepage}}
\lhead{\begin{footnotesize}Functional Programming SS 2019\end{footnotesize}}
\rhead{\begin{footnotesize}379455, 402403, 389343, 402372\end{footnotesize}}
\lfoot{\begin{footnotesize} 
\end{footnotesize}}
\renewcommand{\headrulewidth}{1pt}
\renewcommand{\footrulewidth}{1 pt} 

\begin{document}
\setlength{\parindent}{0em} 


\begin{center} 
\textbf{\huge{Functional Programming} \\ \large{ Excercise Sheet 4}} % Nummerierung anpassen

~\\
Emilie Hastrup-Kiil (379455), 
Julian Schacht (402403), \\
Niklas Gruhn (389343), 
Maximilian Loose (402372)
\end{center}
\textbf{Excercise 1} \\ % Nummerierung anpassen
a) 
$f_{plus} : \mathbb{Z}_{\bot} \times \mathbb{Z}_{\bot} \rightarrow \mathbb{Z}_{\bot}, f_{plus}(x,y) = \begin{cases} 
y & x=0\\ 
x & y=0 \\ 
x+ y &x,y\in \mathbb{Z}\\
\bot & otherwise
\end{cases}$
~\\ \\
b) We will show that $f_{Plus}$ is strict, i.e. if $x_i=\bot$ for $1\leq i\leq2$ then $f_{Plus}(x_1,x_2) = \bot$.\\ \\
Proof:\\
case 1: $x_1 = 0$ and $x_2=\bot$. It follow: $f_{Plus}(x_1,x_2) = x_2 = \bot$ since $x_1= 0 $\\
case 1: $x_1 \neq 0$ and $x_2=\bot$. It follow: $f_{Plus}(x_1,x_2) = \bot$ since $x_2�\notin \mathbb{Z}$ and $x_1 \neq 0 $\\
case 1: $x_2 = 0$ and $x_1=\bot$. It follow: $f_{Plus}(x_1,x_2) = x_1 = \bot$ since $x_2= 0 $\\
case 1: $x_2 \neq 0$ and $x_1=\bot$. It follow: $f_{Plus}(x_1,x_2) =  \bot$ since $x_1�\notin \mathbb{Z} $ and $x_2 \neq 0 $ \\
~\\
c) We will show that $f_{Plus}$ is monotonic, i.e. if  $d \sqsubseteq_{\mathbb{Z}_{\bot} \times \mathbb{Z}_{\bot}} d'$ then $f_{Plus}(d) \sqsubseteq_{\mathbb{Z}_{\bot}} f_{Plus}(d')$.\\ \\
Proof:\\
Let $d \sqsubseteq_{\mathbb{Z}_{\bot} \times \mathbb{Z}_{\bot}} d'$ with $d,d' \in \mathbb{Z}_{\bot} \times \mathbb{Z}_{\bot}$. Then either $d=d'$ or d is less defined than d'. \\If $d=d'$ then $f_{Plus}(d)={Plus}(d')$, thus $f_{Plus}(d) \sqsubseteq {Plus}(d')$.\\
Otherwise $d\neq d'$. Let's say $d=(d_1,d_2)$ and $d'=(d_1',d_2')$ with $ (d_1,d_2) \sqsubseteq (d_1',d_2')$. Then there exists an index $1 \leq i \leq 2$ with $d_i \neq \bot, d_i' \in \mathbb{Z}$ as well as $j\neq i$ with $d_j = d_j'$. Since $f_{Plus}$ is strict $f_{Plus}(d)= \bot$ because $d_i = \bot$ for $1 \leq i \leq 2$.\\ \\ 
Case 1: $d_j = d_j' = \bot$ then we know from the strictness of $f_{Plus}$ that $f(d') = \bot$.� $f_{Plus}(d) = \bot \sqsubseteq_{\mathbb{Z}_{\bot}} \bot = f_{Plus}(d')$ holds.\\ \\
Case 2: $d_j = d_j' \neq \bot$ with $d_j,d_j' \in \mathbb{Z}$ then $d' \in \mathbb{Z}_{\bot} \times \mathbb{Z}_{\bot}$ and $f(d') = a\in \mathbb{Z}$. a is more defined than $\bot$, thus $f_{Plus}(d) = \bot \sqsubseteq_{\mathbb{Z}_{\bot}}a= f_{Plus}(d'), a\in \mathbb{Z}$ holds. \\
~\\
\textbf{Excercise 2} \\% Nummerierung anpassen
~\\
\textbf{Excercise 3} \\% Nummerierung anpassen
~\\     
\textbf{Excercise 4} \\% Nummerierung anpassen

\end{document}
