\documentclass[a4paper,12pt,oneside]{book}

\usepackage[ngerman]{babel}
\usepackage{color}
\usepackage{amsmath}
\usepackage{amsfonts}
\usepackage{fancyhdr}
\usepackage{fancyvrb}
\usepackage{stmaryrd}
\DefineVerbatimEnvironment{code}{Verbatim}{fontsize=\small}
\DefineVerbatimEnvironment{example}{Verbatim}{fontsize=\small}
\usepackage{listings}
\newcommand{\ignore}[1]{}

\usepackage{tikz}
\usetikzlibrary{shapes}
\usetikzlibrary{positioning}
\tikzstyle{data}=[draw,text centered]
\usetikzlibrary{arrows,positioning, calc}
\tikzstyle{vertex}=[draw,circle,minimum size=18pt,inner sep=0pt]

\fboxrule0.3mm
\definecolor{rahmen}{gray}{.3}      % Dunkelgrauer Rahmen
\definecolor{grund}{gray}{.85}         % Hellgrauer Hintergrund

\lstset{ 
    language=Haskell, % choose the language of the code
    basicstyle=\fontfamily{pcr}\selectfont\footnotesize\color{black},
    keywordstyle=\color{black}\mdseries, % style for keywords
    numbers=none, % where to put the line-numbers
    numberstyle=\tiny, % the size of the fonts that are used for the line-numbers     
    backgroundcolor=\color{grund},
    showspaces=false, % show spaces adding particular underscores
    showstringspaces=false, % underline spaces within strings
    showtabs=false, % show tabs within strings adding particular underscores
    frame=single, % adds a frame around the code
    tabsize=2, % sets default tabsize to 2 spaces
    rulesepcolor=\color{rahmen},
    rulecolor=\color{rahmen},
    captionpos=b, % sets the caption-position to bottom
    breaklines=true, % sets automatic line breaking
    breakatwhitespace=false, 
}
 
\pagestyle{fancy}
\usepackage[left=30mm,right=30mm, top=27mm, bottom=22mm]{geometry}

\fancyhf{}
\rfoot{ \textbf{Page \thepage}}
\lhead{\begin{footnotesize}Functional Programming SS 2019\end{footnotesize}}
\rhead{\begin{footnotesize}379455, 402403, 389343, 402372\end{footnotesize}}
\lfoot{\begin{footnotesize} 
\end{footnotesize}}
\renewcommand{\headrulewidth}{1pt}
\renewcommand{\footrulewidth}{1 pt} 

\begin{document}
\setlength{\parindent}{0em} 


\begin{center} 
\textbf{\huge{Functional Programming} \\ \large{ Excercise Sheet 4}} % Nummerierung anpassen

~\\
Emilie Hastrup-Kiil (379455), 
Julian Schacht (402403), \\
Niklas Gruhn (389343), 
Maximilian Loose (402372)
\end{center}
\textbf{Excercise 1} \\ % Nummerierung anpassen
a) 
$f_{plus} : \mathbb{Z}_{\bot} \times \mathbb{Z}_{\bot} \rightarrow \mathbb{Z}_{\bot}, f_{plus}(x,y) = \begin{cases} 
y & x=0\\ 
x & y=0 \\ 
x+ y &x,y\in \mathbb{Z}\\
\bot & otherwise
\end{cases}$
~\\ \\
b) We will show that $f_{Plus}$ is strict, i.e. if $x_i=\bot$ for $1\leq i\leq2$ then $f_{Plus}(x_1,x_2) = \bot$.\\ \\
Proof:\\
case 1: $x_1 = 0$ and $x_2=\bot$. It follow: $f_{Plus}(x_1,x_2) = x_2 = \bot$ since $x_1= 0 $\\
case 1: $x_1 \neq 0$ and $x_2=\bot$. It follow: $f_{Plus}(x_1,x_2) = \bot$ since $x_2�\notin \mathbb{Z}$ and $x_1 \neq 0 $\\
case 1: $x_2 = 0$ and $x_1=\bot$. It follow: $f_{Plus}(x_1,x_2) = x_1 = \bot$ since $x_2= 0 $\\
case 1: $x_2 \neq 0$ and $x_1=\bot$. It follow: $f_{Plus}(x_1,x_2) =  \bot$ since $x_1�\notin \mathbb{Z} $ and $x_2 \neq 0 $ \\
~\\
c) We will show that $f_{Plus}$ is monotonic, i.e. if  $d \sqsubseteq_{\mathbb{Z}_{\bot} \times \mathbb{Z}_{\bot}} d'$ then $f_{Plus}(d) \sqsubseteq_{\mathbb{Z}_{\bot}} f_{Plus}(d')$.\\ \\
Proof:\\
Let $d \sqsubseteq_{\mathbb{Z}_{\bot} \times \mathbb{Z}_{\bot}} d'$ with $d,d' \in \mathbb{Z}_{\bot} \times \mathbb{Z}_{\bot}$. Then either $d=d'$ or d is less defined than d'. \\If $d=d'$ then $f_{Plus}(d)={Plus}(d')$, thus $f_{Plus}(d) \sqsubseteq {Plus}(d')$.\\
Otherwise $d\neq d'$. Let's say $d=(d_1,d_2)$ and $d'=(d_1',d_2')$ with $ (d_1,d_2) \sqsubseteq (d_1',d_2')$. Then there exists an index $1 \leq i \leq 2$ with $d_i \neq \bot, d_i' \in \mathbb{Z}$ as well as $j\neq i$ with $d_j = d_j'$. Since $f_{Plus}$ is strict $f_{Plus}(d)= \bot$ because $d_i = \bot$ for $1 \leq i \leq 2$.\\ \\ 
Case 1: $d_j = d_j' = \bot$ then we know from the strictness of $f_{Plus}$ that $f(d') = \bot$.� $f_{Plus}(d) = \bot \sqsubseteq_{\mathbb{Z}_{\bot}} \bot = f_{Plus}(d')$ holds.\\ \\
Case 2: $d_j = d_j' \neq \bot$ with $d_j,d_j' \in \mathbb{Z}$ then $d' \in \mathbb{Z}_{\bot} \times \mathbb{Z}_{\bot}$ and $f(d') = a\in \mathbb{Z}$. a is more defined than $\bot$, thus $f_{Plus}(d) = \bot \sqsubseteq_{\mathbb{Z}_{\bot}}a= f_{Plus}(d'), a\in \mathbb{Z}$ holds. \\
~\\
\textbf{Excercise 2} \\ \\% Nummerierung anpassen
a) \\ $-': \mathbb{Z}_{\bot} \rightarrow \mathbb{Z}_{\bot}, -'(x) = \begin{cases} 
-x & x\in \mathbb{Z}\\ 
\bot & otherwise
\end{cases}$ \\
b)\\ $*':\mathbb{N}_{\bot} \times \mathbb{N}_{\bot} \rightarrow \mathbb{N}_{\bot} , *'(x,y) = \begin{cases} 
x*y & x,y \in \mathbb{N}\\ 
\bot & otherwise
\end{cases}$�\\�\\
$max:\mathbb{N}_{\bot} \times \mathbb{N}_{\bot} \rightarrow \mathbb{N}_{\bot} , max'(x,y) = \begin{cases} 
x & x > y\\ 
y & y \geq x\\ 
\bot & otherwise
\end{cases}$ \\
This is not monotonic is it?
$max:\mathbb{N}_{\bot} \times \mathbb{N}_{\bot} \rightarrow \mathbb{N}_{\bot} , max'(x,y) = \begin{cases} 
x & x > y\\ 
y & otherwise\\ 
\end{cases}$


~\\
\textbf{Excercise 3} \\ \\% Nummerierung anpassen
a) To show: If $\sqsubseteq_{D_1 \rightarrow D_2}$ is complete on $D_1 \rightarrow D_2$ then $\bot_{D_2}$ exists. \\�\\
Proof:\\
Let $\sqsubseteq_{D_1 \rightarrow D_2}$ be complete on $D_1 \rightarrow D_2$, then $D_1 \rightarrow D_2$ has a smallest element with respect to $\sqsubseteq_{D_1 \rightarrow D_2}$. This element is denoted $\bot_{D_1 \rightarrow D_2}$.
Since $\bot_{D_1 \rightarrow D_2}$ is the smallest element  $\bot_{D_1 \rightarrow D_2} \sqsubseteq f$ holds for every $f:D_1 \rightarrow D_2$ i.e. $\bot_{D_1 \rightarrow D_2}(d) \sqsubseteq f(d)$ holds for every $d \in D_1$. \\
We say that $\bot_{D_1 \rightarrow D_2}(d) = a\in D_2$ and $\bot_{D_1 \rightarrow D_2}(d') = b\in D_2$. Then $a \sqsubseteq f(d)$ and $b \sqsubseteq f(d')$ with $d,d'\in D_1$. Now if $a�\sqsubseteq b$ then a function g that maps every element of $D_1$ to a would exists with $g\sqsubseteq \bot_{D_1 \rightarrow D_2} \sqsubseteq f$ and $a \sqsubseteq f(d)$, $a \sqsubseteq b \sqsubseteq f(d')$ for every $f:D_1 \rightarrow D_2$. $\lightning$ $\bot_{D_1 \rightarrow D_2}$ is the smallest element! Therefore, $a\sqsubseteq b$ and similarily $b\sqsubseteq a$ only holds iff $a = b$. \\
With that we know that $\bot_{D_1 \rightarrow D_2}$ is a constant function.\\
It follows from $\bot_{D_1 \rightarrow D_2}(d) = a \sqsubseteq f(d)$ (for every $d \in D_1$) that $ a \sqsubseteq x$ for every $x \in D_2$. Hence, $D_2$ must have a smallest element a with respect to $\sqsubseteq_{D_2}$ which we denote as $\bot_{D_2}$. \\�\\
b) To show: If $\sqsubseteq_{D_1 \rightarrow D_2}$ is complete on $D_1 \rightarrow D_2$, then for all chains S on $D_2$ the least upper bound $\sqcup S$ of S exists in $D_2$. \\ \\
Proof: \\
Let $\sqsubseteq_{D_1 \rightarrow D_2}$ be complete on $D_1 \rightarrow D_2$, then for every chain S of $D_1 \rightarrow D_2$ there exists a least upper bound $\sqcup S \in D_1 \rightarrow D_2$. For functions $f,f' \in S $ either $f\sqsubseteq f'$ or $f'\sqsubseteq f$ holds, i.e. $f(d)\sqsubseteq f'(d)$ or $f'(d)\sqsubseteq f(d)$ holds for every $d\in D_1$. It follows that for every $d \in D_1$ $S_d = \{ f(d) \vert f�\in S\}$ is a chain in $D_2$. \\From Lemma 2.1.11 b) follows that every chain $S_d$ has a least upper bound $\sqcup S_d$ since S has a least upper bound $\sqcup S$ ($\sqsubseteq_{D_1 \rightarrow D_2}$ is complete). Furthermore, $\sqcup S(d) = \sqcup S_d$.
~\\     
\textbf{Excercise 4} \\% Nummerierung anpassen

\end{document}
